% Options for packages loaded elsewhere
% Options for packages loaded elsewhere
\PassOptionsToPackage{unicode}{hyperref}
\PassOptionsToPackage{hyphens}{url}
\PassOptionsToPackage{dvipsnames,svgnames,x11names}{xcolor}
%
\documentclass[
  portuguese,
  letterpaper,
  DIV=11,
  numbers=noendperiod]{scrartcl}
\usepackage{xcolor}
\usepackage{amsmath,amssymb}
\setcounter{secnumdepth}{5}
\usepackage{iftex}
\ifPDFTeX
  \usepackage[T1]{fontenc}
  \usepackage[utf8]{inputenc}
  \usepackage{textcomp} % provide euro and other symbols
\else % if luatex or xetex
  \usepackage{unicode-math} % this also loads fontspec
  \defaultfontfeatures{Scale=MatchLowercase}
  \defaultfontfeatures[\rmfamily]{Ligatures=TeX,Scale=1}
\fi
\usepackage{lmodern}
\ifPDFTeX\else
  % xetex/luatex font selection
\fi
% Use upquote if available, for straight quotes in verbatim environments
\IfFileExists{upquote.sty}{\usepackage{upquote}}{}
\IfFileExists{microtype.sty}{% use microtype if available
  \usepackage[]{microtype}
  \UseMicrotypeSet[protrusion]{basicmath} % disable protrusion for tt fonts
}{}
\makeatletter
\@ifundefined{KOMAClassName}{% if non-KOMA class
  \IfFileExists{parskip.sty}{%
    \usepackage{parskip}
  }{% else
    \setlength{\parindent}{0pt}
    \setlength{\parskip}{6pt plus 2pt minus 1pt}}
}{% if KOMA class
  \KOMAoptions{parskip=half}}
\makeatother
% Make \paragraph and \subparagraph free-standing
\makeatletter
\ifx\paragraph\undefined\else
  \let\oldparagraph\paragraph
  \renewcommand{\paragraph}{
    \@ifstar
      \xxxParagraphStar
      \xxxParagraphNoStar
  }
  \newcommand{\xxxParagraphStar}[1]{\oldparagraph*{#1}\mbox{}}
  \newcommand{\xxxParagraphNoStar}[1]{\oldparagraph{#1}\mbox{}}
\fi
\ifx\subparagraph\undefined\else
  \let\oldsubparagraph\subparagraph
  \renewcommand{\subparagraph}{
    \@ifstar
      \xxxSubParagraphStar
      \xxxSubParagraphNoStar
  }
  \newcommand{\xxxSubParagraphStar}[1]{\oldsubparagraph*{#1}\mbox{}}
  \newcommand{\xxxSubParagraphNoStar}[1]{\oldsubparagraph{#1}\mbox{}}
\fi
\makeatother

\usepackage{color}
\usepackage{fancyvrb}
\newcommand{\VerbBar}{|}
\newcommand{\VERB}{\Verb[commandchars=\\\{\}]}
\DefineVerbatimEnvironment{Highlighting}{Verbatim}{commandchars=\\\{\}}
% Add ',fontsize=\small' for more characters per line
\usepackage{framed}
\definecolor{shadecolor}{RGB}{241,243,245}
\newenvironment{Shaded}{\begin{snugshade}}{\end{snugshade}}
\newcommand{\AlertTok}[1]{\textcolor[rgb]{0.68,0.00,0.00}{#1}}
\newcommand{\AnnotationTok}[1]{\textcolor[rgb]{0.37,0.37,0.37}{#1}}
\newcommand{\AttributeTok}[1]{\textcolor[rgb]{0.40,0.45,0.13}{#1}}
\newcommand{\BaseNTok}[1]{\textcolor[rgb]{0.68,0.00,0.00}{#1}}
\newcommand{\BuiltInTok}[1]{\textcolor[rgb]{0.00,0.23,0.31}{#1}}
\newcommand{\CharTok}[1]{\textcolor[rgb]{0.13,0.47,0.30}{#1}}
\newcommand{\CommentTok}[1]{\textcolor[rgb]{0.37,0.37,0.37}{#1}}
\newcommand{\CommentVarTok}[1]{\textcolor[rgb]{0.37,0.37,0.37}{\textit{#1}}}
\newcommand{\ConstantTok}[1]{\textcolor[rgb]{0.56,0.35,0.01}{#1}}
\newcommand{\ControlFlowTok}[1]{\textcolor[rgb]{0.00,0.23,0.31}{\textbf{#1}}}
\newcommand{\DataTypeTok}[1]{\textcolor[rgb]{0.68,0.00,0.00}{#1}}
\newcommand{\DecValTok}[1]{\textcolor[rgb]{0.68,0.00,0.00}{#1}}
\newcommand{\DocumentationTok}[1]{\textcolor[rgb]{0.37,0.37,0.37}{\textit{#1}}}
\newcommand{\ErrorTok}[1]{\textcolor[rgb]{0.68,0.00,0.00}{#1}}
\newcommand{\ExtensionTok}[1]{\textcolor[rgb]{0.00,0.23,0.31}{#1}}
\newcommand{\FloatTok}[1]{\textcolor[rgb]{0.68,0.00,0.00}{#1}}
\newcommand{\FunctionTok}[1]{\textcolor[rgb]{0.28,0.35,0.67}{#1}}
\newcommand{\ImportTok}[1]{\textcolor[rgb]{0.00,0.46,0.62}{#1}}
\newcommand{\InformationTok}[1]{\textcolor[rgb]{0.37,0.37,0.37}{#1}}
\newcommand{\KeywordTok}[1]{\textcolor[rgb]{0.00,0.23,0.31}{\textbf{#1}}}
\newcommand{\NormalTok}[1]{\textcolor[rgb]{0.00,0.23,0.31}{#1}}
\newcommand{\OperatorTok}[1]{\textcolor[rgb]{0.37,0.37,0.37}{#1}}
\newcommand{\OtherTok}[1]{\textcolor[rgb]{0.00,0.23,0.31}{#1}}
\newcommand{\PreprocessorTok}[1]{\textcolor[rgb]{0.68,0.00,0.00}{#1}}
\newcommand{\RegionMarkerTok}[1]{\textcolor[rgb]{0.00,0.23,0.31}{#1}}
\newcommand{\SpecialCharTok}[1]{\textcolor[rgb]{0.37,0.37,0.37}{#1}}
\newcommand{\SpecialStringTok}[1]{\textcolor[rgb]{0.13,0.47,0.30}{#1}}
\newcommand{\StringTok}[1]{\textcolor[rgb]{0.13,0.47,0.30}{#1}}
\newcommand{\VariableTok}[1]{\textcolor[rgb]{0.07,0.07,0.07}{#1}}
\newcommand{\VerbatimStringTok}[1]{\textcolor[rgb]{0.13,0.47,0.30}{#1}}
\newcommand{\WarningTok}[1]{\textcolor[rgb]{0.37,0.37,0.37}{\textit{#1}}}

\usepackage{longtable,booktabs,array}
\usepackage{calc} % for calculating minipage widths
% Correct order of tables after \paragraph or \subparagraph
\usepackage{etoolbox}
\makeatletter
\patchcmd\longtable{\par}{\if@noskipsec\mbox{}\fi\par}{}{}
\makeatother
% Allow footnotes in longtable head/foot
\IfFileExists{footnotehyper.sty}{\usepackage{footnotehyper}}{\usepackage{footnote}}
\makesavenoteenv{longtable}
\usepackage{graphicx}
\makeatletter
\newsavebox\pandoc@box
\newcommand*\pandocbounded[1]{% scales image to fit in text height/width
  \sbox\pandoc@box{#1}%
  \Gscale@div\@tempa{\textheight}{\dimexpr\ht\pandoc@box+\dp\pandoc@box\relax}%
  \Gscale@div\@tempb{\linewidth}{\wd\pandoc@box}%
  \ifdim\@tempb\p@<\@tempa\p@\let\@tempa\@tempb\fi% select the smaller of both
  \ifdim\@tempa\p@<\p@\scalebox{\@tempa}{\usebox\pandoc@box}%
  \else\usebox{\pandoc@box}%
  \fi%
}
% Set default figure placement to htbp
\def\fps@figure{htbp}
\makeatother



\ifLuaTeX
\usepackage[bidi=basic]{babel}
\else
\usepackage[bidi=default]{babel}
\fi
% get rid of language-specific shorthands (see #6817):
\let\LanguageShortHands\languageshorthands
\def\languageshorthands#1{}


\setlength{\emergencystretch}{3em} % prevent overfull lines

\providecommand{\tightlist}{%
  \setlength{\itemsep}{0pt}\setlength{\parskip}{0pt}}



 


\KOMAoption{captions}{tableheading}
\makeatletter
\@ifpackageloaded{caption}{}{\usepackage{caption}}
\AtBeginDocument{%
\ifdefined\contentsname
  \renewcommand*\contentsname{Índice}
\else
  \newcommand\contentsname{Índice}
\fi
\ifdefined\listfigurename
  \renewcommand*\listfigurename{Lista de Figuras}
\else
  \newcommand\listfigurename{Lista de Figuras}
\fi
\ifdefined\listtablename
  \renewcommand*\listtablename{Lista de Tabelas}
\else
  \newcommand\listtablename{Lista de Tabelas}
\fi
\ifdefined\figurename
  \renewcommand*\figurename{Figura}
\else
  \newcommand\figurename{Figura}
\fi
\ifdefined\tablename
  \renewcommand*\tablename{Tabela}
\else
  \newcommand\tablename{Tabela}
\fi
}
\@ifpackageloaded{float}{}{\usepackage{float}}
\floatstyle{ruled}
\@ifundefined{c@chapter}{\newfloat{codelisting}{h}{lop}}{\newfloat{codelisting}{h}{lop}[chapter]}
\floatname{codelisting}{Listagem}
\newcommand*\listoflistings{\listof{codelisting}{Lista de Listagens}}
\makeatother
\makeatletter
\makeatother
\makeatletter
\@ifpackageloaded{caption}{}{\usepackage{caption}}
\@ifpackageloaded{subcaption}{}{\usepackage{subcaption}}
\makeatother
\usepackage{bookmark}
\IfFileExists{xurl.sty}{\usepackage{xurl}}{} % add URL line breaks if available
\urlstyle{same}
\hypersetup{
  pdftitle={JSPlotly e GSPlotly na Escola},
  pdflang={pt},
  colorlinks=true,
  linkcolor={blue},
  filecolor={Maroon},
  citecolor={Blue},
  urlcolor={Blue},
  pdfcreator={LaTeX via pandoc}}


\title{JSPlotly e GSPlotly na Escola}
\author{}
\date{}
\begin{document}
\maketitle

\renewcommand*\contentsname{Índice}
{
\hypersetup{linkcolor=}
\setcounter{tocdepth}{2}
\tableofcontents
}

~~Para ilustrar o potencial de uso do \emph{JSPlotly} para o ensino
fundamental e médio, seguem alguns exemplos de simulações e cujos
gráficos são frequentemente encontrados em livros-texto e conteúdos
afins. Para um melhor aproveitamento de cada tema, experimente seguir as
sugestões para \emph{manipulação paramétrica} em cada tema.

\hfill\break

\section*{Instruções}\label{instruuxe7uxf5es}
\addcontentsline{toc}{section}{Instruções}

\begin{Shaded}
\begin{Highlighting}[]
\FloatTok{1.}\NormalTok{ Escolha um tema;}
\FloatTok{2.}\NormalTok{ Clique no gráfico correspondente;}
\FloatTok{3.}\NormalTok{ Clique em }\StringTok{"Add Plot"}\NormalTok{;}
\FloatTok{4.}\NormalTok{ Use o mouse para interatividade e}\SpecialCharTok{/}\NormalTok{ou edite o código. }

\NormalTok{Lembrete}\SpecialCharTok{:}\NormalTok{ o editor usa desfazer}\SpecialCharTok{/}\NormalTok{refazer infinitos no código (Ctrl}\SpecialCharTok{+}\NormalTok{Z }\SpecialCharTok{/}\NormalTok{ Shift}\SpecialCharTok{+}\NormalTok{Ctrl}\SpecialCharTok{+}\NormalTok{Z) }\SpecialCharTok{!}
\end{Highlighting}
\end{Shaded}

\hfill\break

\section{Matemática}\label{matemuxe1tica}

\subsection{Contexto - Trigonometria (EM13MAT306, EM13MAT308,
EM13MAT307)}\label{contexto---trigonometria-em13mat306-em13mat308-em13mat307}

~~A simulação a seguir objetiva facilitar a visualização para alguns
conceitos em trigonometria, \emph{seno, cosseno e tangente}. O código
permite usar um \emph{menu suspenso} para cada função trigonométrica,
bem como um \emph{slider} para alterar a frequência em radianos.

\subsection*{Equação:}\label{equauxe7uxe3o}
\addcontentsline{toc}{subsection}{Equação:}

\textbf{1. Função seno:}

\[
y = \sin(\omega x)
\]

\textbf{2. Função cosseno:}

\[
y = \cos(\omega x)
\]

\textbf{3. Função tangente:}

\[
y = \tan(\omega x)
\]\\

\href{Eq/jsp_trigonom2.html}{\pandocbounded{\includegraphics[keepaspectratio]{Eq/trigonom.png}}}

\subsection*{Sugestão:}\label{sugestuxe3o}
\addcontentsline{toc}{subsection}{Sugestão:}

\begin{Shaded}
\begin{Highlighting}[]
\FloatTok{1.}\NormalTok{ Selecione, alternativamente, a função seno, cosseno, e tangente, utilizando}\SpecialCharTok{{-}}\NormalTok{se o }\StringTok{"menu suspenso"}\NormalTok{;}
\FloatTok{2.}\NormalTok{ Experimente o efeito de se alterar a frequência da função na barra }\FunctionTok{deslizante}\NormalTok{ (}\StringTok{"slider"}\NormalTok{);}
\FloatTok{3.}\NormalTok{ Sobreponha um gráfico de seno e um de cosseno para observar suas diferenças;}
\FloatTok{4.}\NormalTok{ Repita o mesmo para o gráfico de tangente.}
\end{Highlighting}
\end{Shaded}

\section{Matemática Financeira}\label{matemuxe1tica-financeira}

\subsection{Contexto - Juros Compostos
(EM13MAT402):}\label{contexto---juros-compostos-em13mat402}

~~Também conhecido pela máxima \emph{``juros sobre juros''}, os juros
compostos incorporam valor ao capital ao longo do tempo, resultando no
cresimento do montante final.

\subsection*{Equação:}\label{equauxe7uxe3o-1}
\addcontentsline{toc}{subsection}{Equação:}

\[
M = C \cdot (1 + i)^t
\]

\emph{Onde,}

\begin{itemize}
\tightlist
\item
  M: montante final
\item
  C: capital inicial
\item
  i: taxa de juros por período (em decimal)
\item
  t: número de períodos (ex: meses)
\end{itemize}

\href{Eq/jsp_juros2.html}{\pandocbounded{\includegraphics[keepaspectratio]{Eq/juros.png}}}

\subsection*{Sugestão:}\label{sugestuxe3o-1}
\addcontentsline{toc}{subsection}{Sugestão:}

\begin{Shaded}
\begin{Highlighting}[]
\FloatTok{1.}\NormalTok{ Varie o período de contratação, a taxa mensal de juros ou o montante inicial.}
\FloatTok{2.}\NormalTok{ Experimente combinar os parâmetros na variação.}
\FloatTok{3.}\NormalTok{ Avalie a diferença visual entre um investimento e um empréstimo, inserido valor positivo de capital inicial para o }\DecValTok{1}\NormalTok{o. e negativo para o }\DecValTok{2}\NormalTok{o. }
\FloatTok{4.}\NormalTok{ Observe a curva descendente para um empréstimo simulado com capital inicial negativo. Os valores remanescentes referem}\SpecialCharTok{{-}}\NormalTok{se à dívida faltante para quitar o empréstimo.}
\end{Highlighting}
\end{Shaded}

\section{Estatística}\label{estatuxedstica}

\subsection{Contexto - Curva de distribuição normal (EM13MAT316,
EM13MAT407,
EM13MAT312)}\label{contexto---curva-de-distribuiuxe7uxe3o-normal-em13mat316-em13mat407-em13mat312}

~~~~~~Amostragem e população são temas comuns para observação de dados
em procedimentos analíticos. As curvas de distribuição estatísticas para
isso envolvem a distribuição \emph{t-Student}, \emph{F-Snedecor},
\emph{Chi-quadrado}, e \emph{distribuição normal}. A curva de
distribuição normal reflete o comportamento estatístico para fenômenos
naturais em uma dada uma população de dados.

~~~~~~O exemplo pretende ilustrar o uso da transformação \emph{z}, o
cálculo de valores críticos, e a interpretação da área sob a curva no
estudo da distribuição normal e de inferência estatística.

\subsection*{Equação}\label{equauxe7uxe3o-2}
\addcontentsline{toc}{subsection}{Equação}

~~~~~~A função densidade da distribuição normal (ou Gaussiana) é dada
abaixo?

\[
f(x) = \frac{1}{\sigma \sqrt{2\pi}} \, e^{ -\frac{(x - \mu)^2}{2\sigma^2} }
\]

\emph{Onde:}

\begin{itemize}
\tightlist
\item
  \(\mu\) = 0 (média da distribuição);
\item
  \(\sigma\) = 1 (desvio padrão);
\item
  \emph{x} = variável aleatória contínua; \emph{f} = função de densidade
  da distribuição normal
\end{itemize}

~~~~~~Da equação acima pode-se extrair \emph{z}, o valor da variável
aleatória padronizada para média nula e desvio-padrão unitário,
representando o valor no eixo das abscissas:

\[
z = \frac{x - \mu}{\sigma}
\]

\href{Eq/jsp_estat2.html}{\pandocbounded{\includegraphics[keepaspectratio]{Eq/estat.png}}}

\subsection*{Sugestão:}\label{sugestuxe3o-2}
\addcontentsline{toc}{subsection}{Sugestão:}

\begin{Shaded}
\begin{Highlighting}[]
\FloatTok{1.}\NormalTok{ Experimente alterar o valor de }\StringTok{"p"}\NormalTok{ e rodar o gráfico. Esse valor representa a probabilidade de se observar, sob a hipótese nula, um valor tão extremo ou mais extremo do que o observado — ou seja, mede a evidência contra a hipótese nula. No gráfico, representa a área sob a curva normal nas regiões críticas, indicando a chance de ocorrência do resultado observado por puro acaso.}
\end{Highlighting}
\end{Shaded}

\section{Física}\label{fuxedsica}

\subsection{Contexto - Energia potencial elástica
(EM13CNT102,EM13CNT202,
EM13MAT402)}\label{contexto---energia-potencial-eluxe1stica-em13cnt102em13cnt202-em13mat402}

~~A deformação de um material elástico é diretamente proporcional à
força exercida sobre esse, e limitada às propriedades do material.

\subsection*{Equação}\label{equauxe7uxe3o-3}
\addcontentsline{toc}{subsection}{Equação}

~~O comportamento de uma mola ideal é descrita pela \emph{Lei de Hooke}
abaixo:

\[
F = -k*x
\]

\emph{Onde:}

\begin{itemize}
\tightlist
\item
  F = força restauradora da mola (N);
\item
  k = constante elástica da mola (N/m);
\item
  x = deformação (m).
\end{itemize}

~~Por outro lado, a \emph{energia potencial elástica} envolvida é
descrita pela relação quadrática que segue:

\[
E = \frac{1}{2}*k*x^2
\]

\emph{Onde:}

\begin{itemize}
\tightlist
\item
  E = energia potencial elástica (J).
\end{itemize}

\href{Eq/jsp_fisica2.html}{\pandocbounded{\includegraphics[keepaspectratio]{Eq/fisica2.png}}}

\subsection*{Sugestão:}\label{sugestuxe3o-3}
\addcontentsline{toc}{subsection}{Sugestão:}

\begin{Shaded}
\begin{Highlighting}[]
\FloatTok{1.}\NormalTok{ Experimente alterar o valor da constante elástica da mola para evidenciar seu efeito, relacionando}\SpecialCharTok{{-}}\NormalTok{a com molas mais rígidas ou menos rígidas;}
\FloatTok{2.}\NormalTok{ Altere os limites de deformação da mola na }\StringTok{"estrutura de constrole"}\NormalTok{ do código (ex}\SpecialCharTok{:} \StringTok{"for (let x = {-}0.7)"}\NormalTok{), e observe o efeito na energia potencial máxima;}
\FloatTok{3.}\NormalTok{ Observe que, pela operação quadrática no valor da deformação, a energia potencial é sempre positiva.}
\end{Highlighting}
\end{Shaded}

\subsection{Contexto - Movimento de corpos
(EM13CNT102):}\label{contexto---movimento-de-corpos-em13cnt102}

~~O código a seguir ilustra a trajetória de um lançamento oblíquo com
ângulo ajustável por uma barra deslizante (\emph{slider}), útil para
explorar os conceitos de cinemática.

\subsection*{Equação:}\label{equauxe7uxe3o-4}
\addcontentsline{toc}{subsection}{Equação:}

\textbf{1. Equação geral}

\[
y(x) = x \cdot \tan(\theta) - \frac{g}{2 v_0^2 \cos^2(\theta)} \cdot x^2
\]

\emph{Onde:}

\begin{itemize}
\tightlist
\item
  y(x): altura em função da distância horizontal;
\item
  x: posição horizontal (m);
\item
  \(\theta\): ângulo de lançamento em relação à horizontal (radianos ou
  graus);
\item
  v0: velocidade inicial do projétil (m/s);
\item
  g: aceleração da gravidade (9,8 m/s²\(^{2}\))
\end{itemize}

\textbf{2. Tempo total de vôo:}

\[
t_{\text{total}} = \frac{2 v_0 \sin(\theta)}{g}
\]

\textbf{3. Posição Horizontal ao longo do tempo}

\[
x(t) = v_0 \cos(\theta) \cdot t
\]

\href{Eq/jsp_fis_pedra2.html}{\pandocbounded{\includegraphics[keepaspectratio]{Eq/fis_pedra.png}}}

\subsection*{Sugestão:}\label{sugestuxe3o-4}
\addcontentsline{toc}{subsection}{Sugestão:}

\begin{Shaded}
\begin{Highlighting}[]
\FloatTok{1.}\NormalTok{ Veja que há uma barra deslizante para ângulos iniciais na simulação. Arraste a barra para outro ângulo e adicione o gráfico, comparando o efeito dessa modificação.}
\FloatTok{2.}\NormalTok{ Altere a velocidade inicial no código, e observe o efeito no gráfico.}
\FloatTok{3.}\NormalTok{ Simule uma }\StringTok{"condição lunar"}\NormalTok{ para a trajetória, e cuja gravidade é em torno de }\DecValTok{1}\SpecialCharTok{/}\DecValTok{6}\NormalTok{ a da }\FunctionTok{Terra}\NormalTok{ (}\SpecialCharTok{\textasciitilde{}}\FloatTok{1.6}\NormalTok{ m}\SpecialCharTok{/}\NormalTok{s²).}
\end{Highlighting}
\end{Shaded}

\section{Química}\label{quuxedmica}

\subsection{Contexto - Capacidade Calorífica (EF09CI06, EM13CNT104,
EM13CNT203)}\label{contexto---capacidade-caloruxedfica-ef09ci06-em13cnt104-em13cnt203}

~~A simulação que segue visa observar a relação entre o calor trocado
(\emph{Q}), a massa (\emph{m}), a capacidade calorífica (\emph{c}) e a
variação de temperatura (\emph{\(\Delta\)T}).

\subsection*{Equação:}\label{equauxe7uxe3o-5}
\addcontentsline{toc}{subsection}{Equação:}

\[
Q = c \cdot m \cdot \Delta T
\]\\

\href{Eq/jsp_cp2.html}{\pandocbounded{\includegraphics[keepaspectratio]{Eq/cp.png}}}

\subsection*{Sugestão:}\label{sugestuxe3o-5}
\addcontentsline{toc}{subsection}{Sugestão:}

\begin{Shaded}
\begin{Highlighting}[]
\FloatTok{1.}\NormalTok{ Experimente variar inicialmente a temperatura, sobrepondo alguns gráficos;}
\FloatTok{2.}\NormalTok{ Varie também a massa no editor de códigos, para comparação.}
\end{Highlighting}
\end{Shaded}

\subsection{Contexto: Mistura de substâncias em reação exotérmica -
gráfico 3D (EF09CI02, EM13CNT103,
EM13CNT103)}\label{contexto-mistura-de-substuxe2ncias-em-reauxe7uxe3o-exotuxe9rmica---gruxe1fico-3d-ef09ci02-em13cnt103-em13cnt103}

~~Simulações podem ser realizadas sem necessariamente envolver uma
relação matemática, como na observação experimental de duas variáveis,
como tempo e concentração, simulando uma reação química exotérmica.
Segue um exemplo interativo em 3D.

~~Nesse caso, a equação utilizada no editor envolve uma variação suave
de temperatura ao longo do tempo, empregando-se a função seno e uma
temperatura inicial (vide o código).

\hfill\break

\href{Eq/jsp_quim2.html}{\pandocbounded{\includegraphics[keepaspectratio]{Eq/quim2.png}}}

\subsection*{Sugestão:}\label{sugestuxe3o-6}
\addcontentsline{toc}{subsection}{Sugestão:}

\begin{Shaded}
\begin{Highlighting}[]
\FloatTok{1.}\NormalTok{ Experimente variar inicialmente a temperatura, sobrepondo alguns gráficos;}
\FloatTok{2.}\NormalTok{ Varie também a massa no editor de códigos, para comparação.}
\end{Highlighting}
\end{Shaded}

\section{Biologia}\label{biologia}

\subsection{Contexto - Modelo de Crescimento Populacional com Fase Lag
(EM13CNT102)}\label{contexto---modelo-de-crescimento-populacional-com-fase-lag-em13cnt102}

~~Este modelo apresenta uma função logística que simula o crescimento
populacional (microorganismos, células, por ex), acompanhado por um
componente de retardo. Variando-se os parâmetros no editor, é possível
estimar diversos perfis de crescimento populacional.

\subsection*{Equação:}\label{equauxe7uxe3o-6}
\addcontentsline{toc}{subsection}{Equação:}

\[
N(t) = \frac{K}{1 + \left(\frac{K - N_0}{N_0}\right) \cdot e^{-r \cdot A(t) \cdot t}}, \quad \text{com } A(t) = \frac{1}{1 + e^{-k(t - t_0)}}
\]

\emph{Onde:}

\begin{itemize}
\tightlist
\item
  K = capacidade de suporte ambiental;
\item
  N0 = população inicial;
\item
  r = taxa intrínseca de crescimento;
\item
  A(t) = fator de ativação do crescimento com atraso (fase lag);
\item
  t0 = ponto médio de transição entre fase lag e fase log;
\item
  k = constante de suavidade do retardo (fixado como 0.5 no código)
\end{itemize}

\href{Eq/jsp_crescim2.html}{\pandocbounded{\includegraphics[keepaspectratio]{Eq/crescim.png}}}

\subsection*{Sugestão:}\label{sugestuxe3o-7}
\addcontentsline{toc}{subsection}{Sugestão:}

\begin{Shaded}
\begin{Highlighting}[]
\FloatTok{1.}\NormalTok{ Experimente variar os parâmetros da equação, combinando alguns, e comparando seus efeitos sobre os gráficos}\SpecialCharTok{:}
\NormalTok{  a. Capacidade de suporte;}
\NormalTok{  b. População inicial;}
\NormalTok{  c. Taxa de crescimento;}
\NormalTok{  d. }\FunctionTok{Retardo}\NormalTok{ (fase lag);}
\end{Highlighting}
\end{Shaded}

\subsection{Contexto - Eficiência energética e cadeia alimentar
(EF06CI02, EM13CNT202,
EM13CNT203)}\label{contexto---eficiuxeancia-energuxe9tica-e-cadeia-alimentar-ef06ci02-em13cnt202-em13cnt203}

~~~~~~Segue um exemplo para retratar a transferëncia de energia entre
diferentes níveis tróficos de uma cadeia alimentar. Embora não haja
propriamente uma função matemática que a descreva, pode-se aplicar a
\href{https://en.wikipedia.org/wiki/Ecological_efficiency?}{lei dos
10\%} de eficiência ecológica entre os níveis da cadeia, o que resulta
numa relação logaritmica de transferência.

\href{Eq/jsp_bio2.html}{\pandocbounded{\includegraphics[keepaspectratio]{Eq/bio2.png}}}

\subsection*{Sugestão}\label{sugestuxe3o-8}
\addcontentsline{toc}{subsection}{Sugestão}

\begin{Shaded}
\begin{Highlighting}[]
\FloatTok{1.}\NormalTok{ A regra de Lindeman, esboçada na referência acima, estabelece uma variação para }\DecValTok{5{-}20}\NormalTok{\% de eficiência energética no ecossistema. Assim, experimente sobrepor as curvas com tais taxas;}
\FloatTok{2.}\NormalTok{ Se quiser observar a relação logarítmica da transferência de energia, acrescente o comando }\StringTok{"type: \textquotesingle{}log\textquotesingle{},"}\NormalTok{ , logo abaixo de }\StringTok{"title: \textquotesingle{}Energia disponível (unidades)\textquotesingle{},"}\NormalTok{.}
\end{Highlighting}
\end{Shaded}

\section{Geografia}\label{geografia}

\subsection{Contexto: Mapa do Brasil e Capitais (EM13CHS101, EM13CHS202,
EM13CHS301)}\label{contexto-mapa-do-brasil-e-capitais-em13chs101-em13chs202-em13chs301}

~~Não apenas de equações vive o \emph{JSPlotly} ! Com a biblioteca
\emph{Plotly.js} que o compõe, é possível produzir também mapas
interativos, como o da simulação que segue.

\href{Eq/jsp_geogr3.html}{\pandocbounded{\includegraphics[keepaspectratio]{Eq/geogr.png}}}

\subsection*{Sugestão:}\label{sugestuxe3o-9}
\addcontentsline{toc}{subsection}{Sugestão:}

\begin{Shaded}
\begin{Highlighting}[]
\FloatTok{1.}\NormalTok{ Experimente usar o botão de rolagem do mouse e ícone }\StringTok{"pan"}\NormalTok{ da barra superior, para interagir com o mapa.}
\end{Highlighting}
\end{Shaded}

\subsection{Contexto: Mapa e PIB do Oriente Médio (EF09GE03, EF08GE06,
EM13CHS104,
EM13CHS201)}\label{contexto-mapa-e-pib-do-oriente-muxe9dio-ef09ge03-ef08ge06-em13chs104-em13chs201}

\href{Eq/jsp_geogr2.html}{\pandocbounded{\includegraphics[keepaspectratio]{Eq/geogr2.png}}}

\subsection*{Sugestão:}\label{sugestuxe3o-10}
\addcontentsline{toc}{subsection}{Sugestão:}

\begin{Shaded}
\begin{Highlighting}[]
\FloatTok{1.}\NormalTok{ Experimente usar o botão de rolagem do mouse;}
\FloatTok{2.}\NormalTok{ Clique num país para identificar seu nome e produto interno bruto aproximado;;}
\FloatTok{3.}\NormalTok{ Modique o código para atualizar algum dado, ou para modificar a informação (trocando PIB por outro dado, por ex).}
\end{Highlighting}
\end{Shaded}

\section{História}\label{histuxf3ria}

\subsection{Contexto - Distribuição de escravizados nas Américas no
período de 1500--1888 (EF08HI06, EM13CHS104,
EM13CHS503):}\label{contexto---distribuiuxe7uxe3o-de-escravizados-nas-amuxe9ricas-no-peruxedodo-de-15001888-ef08hi06-em13chs104-em13chs503}

~~~Esta simulação apresenta um \emph{gráfico de barras} interativo para
seleção de período por \emph{menu suspenso}, e tangente à quantidade
estimada de africanos escravizados desembarcados nas principais regiões
da América. Os dados são estimativas aproximadas para ilustrar o
potencial de visualização do aplicativo, embora servindo como ponto de
partida para discussões educacionais mais precisas. Fontes passíveis de
consulta incluem
\href{https://www.slavevoyages.org/past/database/african-origins}{Slave
Voyages}.

\hfill\break

\href{Eq/jsp_escravos2.html}{\pandocbounded{\includegraphics[keepaspectratio]{Eq/escravos.png}}}

\subsection*{Sugestão:}\label{sugestuxe3o-11}
\addcontentsline{toc}{subsection}{Sugestão:}

\begin{Shaded}
\begin{Highlighting}[]
\FloatTok{1.}\NormalTok{ Experimente alterar entre os períodos no menu suspenso, comparando as estimativas de tráfico escravo;}
\FloatTok{2.}\NormalTok{ Selecione um período, crie o gráfico, selecione outro período, e adicione outro gráfico. Isso permite comparar o quantitativo de escravos aportados pelas barras duplas formadas, e passagem do mouse sobre cada barra.}
\end{Highlighting}
\end{Shaded}

\subsection{Contexto - Linha do tempo de eventos da Idade Média
(EM13CHS101,
EM13CHS102)}\label{contexto---linha-do-tempo-de-eventos-da-idade-muxe9dia-em13chs101-em13chs102}

\href{Eq/jsp_hist2.html}{\pandocbounded{\includegraphics[keepaspectratio]{Eq/hist2.png}}}
Fonte:
\href{https://www.encyclopedia.com/history/encyclopedias-almanacs-transcripts-and-maps/timeline-events-middle-ages?}{Encyclopedia.com}

\subsection*{Sugestão:}\label{sugestuxe3o-12}
\addcontentsline{toc}{subsection}{Sugestão:}

\begin{Shaded}
\begin{Highlighting}[]
\FloatTok{1.}\NormalTok{ Experimente alterar no código eventos e períodos, e destinados a outro período da História humana.}
\end{Highlighting}
\end{Shaded}

\section{Linguagens}\label{linguagens}

\subsection{Contexto - Gírias faladas no Brasil de 1980-2020 (EF89LP19,
EM13LGG102)}\label{contexto---guxedrias-faladas-no-brasil-de-1980-2020-ef89lp19-em13lgg102}

~~Esta simulação é direcionada para uma estimativa do uso de gírias
faladas no Brasil durante os últimos 40 anos. A representação dá-se num
gráfico de pizza, e a seleção por década num menu suspenso.

\hfill\break

\href{Eq/jsp_linguag_pizza.html}{\pandocbounded{\includegraphics[keepaspectratio]{Eq/linguag_pizza.png}}}

\begin{Shaded}
\begin{Highlighting}[]
\FloatTok{1.}\NormalTok{ Pode}\SpecialCharTok{{-}}\NormalTok{se usar a passagem de mouse para observar o }\StringTok{"tip"}\NormalTok{ (dica) apresentada para cada dado no gráfico.}
\end{Highlighting}
\end{Shaded}

\subsection{Contexto - Frequência de palavras em texto (EM13LGG101,
EM13LGG302,
EM13LGG303)}\label{contexto---frequuxeancia-de-palavras-em-texto-em13lgg101-em13lgg302-em13lgg303}

\href{Eq/jsp_linguag2.html}{\pandocbounded{\includegraphics[keepaspectratio]{Eq/linguag2.png}}}

\subsection*{Sugestão}\label{sugestuxe3o-13}
\addcontentsline{toc}{subsection}{Sugestão}

\begin{Shaded}
\begin{Highlighting}[]
\FloatTok{1.}\NormalTok{ Experimente substituir o texto do código por outro;}
\FloatTok{2.}\NormalTok{ Experimente variar o quantitativo de termos mais frequentes na variável }\StringTok{"const entradas"}\NormalTok{ (opcionalmente, varie também no título do gráfico, para fazer sentido);}
\FloatTok{3.}\NormalTok{ Compare um texto em português com sua tradução para inglês ou outra língua.}
\end{Highlighting}
\end{Shaded}

\section{Diagramas e Fluxogramas}\label{diagramas-e-fluxogramas}

~~O aplicativo também permite a criação de outros objetos didáticos
interativos, sem a necessidade de inserção de equações, como diagramas e
fluxogramas. Seguem exemplos.

\subsection{Contexto: Diagrama de ciclo claro e escuro da fotossíntese
(EM13CNT101, EM13CNT103, EM13CNT201,
EM13MAT405)}\label{contexto-diagrama-de-ciclo-claro-e-escuro-da-fotossuxedntese-em13cnt101-em13cnt103-em13cnt201-em13mat405}

\href{Eq/jsp_diagramaEB2.html}{\pandocbounded{\includegraphics[keepaspectratio]{Eq/diagramaEB.png}}}

\begin{Shaded}
\begin{Highlighting}[]
\FloatTok{1.}\NormalTok{ Experimente reposicionar as entradas e saídas (ex}\SpecialCharTok{:}\NormalTok{ Luz, Glicose) por simples arraste de mouse;}
\FloatTok{2.}\NormalTok{ Substitua os termos dentro dos quadros, ou mude outros aspectos do }\FunctionTok{diagrama}\NormalTok{ (cor, preenchimento, por ex).}
\end{Highlighting}
\end{Shaded}

\subsection{Contexto: Fluxograma para o ciclo da água (EF06CI03,
EF06CI04, EM13CNT103,
EM13CNT202)}\label{contexto-fluxograma-para-o-ciclo-da-uxe1gua-ef06ci03-ef06ci04-em13cnt103-em13cnt202}

\href{Eq/jsp_fluxogramaEB2.html}{\pandocbounded{\includegraphics[keepaspectratio]{Eq/fluxogramaEB.png}}}

\subsection*{Sugestões:}\label{sugestuxf5es}
\addcontentsline{toc}{subsection}{Sugestões:}

\begin{Shaded}
\begin{Highlighting}[]
\FloatTok{1.}\NormalTok{ Como para diagramas acima, experimente alterar no código as propriedades das setas e dos termos e campos envolvidos no fluxograma;}
\FloatTok{2.}\NormalTok{ Substitua termos para a formação de outro fluxograma;}
\FloatTok{3.}\NormalTok{ Reposicione os objetos na área gráfica (campos, termos, setas) com auxílio do mouse.}
\end{Highlighting}
\end{Shaded}

\section{Artes}\label{artes}

\subsection{Contexto - Composição fractal (EM13MAT301, EM13MAT305,
EM13MAT401, EM13ARM502,
EF09MA10)}\label{contexto---composiuxe7uxe3o-fractal-em13mat301-em13mat305-em13mat401-em13arm502-ef09ma10}

~~~~~~Fractais constituem objetos geométricos com simetria de escala,
formando padrões que se repetem em partes menores do objeto original. As
representações matemáticas mais comuns são os fractais de Mandelbrot e
os fractais de Julia.

\subsection*{Equação}\label{equauxe7uxe3o-7}
\addcontentsline{toc}{subsection}{Equação}

~~~~~Fractais de Julia são formados em plano cartesiano complexo, pela
somatória dos componentes real e imaginário. A fórmula básica para o
conjunto de Julia é dada por:

\[
z_{n+1}=z_{n}^{2}+c
\]

\emph{Onde:}

\begin{itemize}
\item
  z ∈ C: geralmente inicializado como o ponto do plano complexo;
\item
  c ∈ C: fixo para cada conjunto de Julia.
\end{itemize}

\href{Eq/jsp_julia2.html}{\pandocbounded{\includegraphics[keepaspectratio]{Eq/julia.png}}}

\subsection*{Sugestão}\label{sugestuxe3o-14}
\addcontentsline{toc}{subsection}{Sugestão}

\begin{Shaded}
\begin{Highlighting}[]
\FloatTok{1.}\NormalTok{ Experimente alterar os componentes Real e Imaginário da fórmula, para obter padrões artísticos distintos. Seguem sugestões}\SpecialCharTok{:}
\NormalTok{  c }\OtherTok{=} \DecValTok{0} \SpecialCharTok{+} \DecValTok{0}\NormalTok{i}
\NormalTok{  c }\OtherTok{=} \SpecialCharTok{{-}}\FloatTok{0.4} \SpecialCharTok{+} \FloatTok{0.6}\NormalTok{i}
\NormalTok{  c }\OtherTok{=} \FloatTok{0.285} \SpecialCharTok{+} \DecValTok{0}\NormalTok{i}
\NormalTok{  c }\OtherTok{=} \SpecialCharTok{{-}}\FloatTok{0.8} \SpecialCharTok{+} \FloatTok{0.156}\NormalTok{i}
\NormalTok{  c }\OtherTok{=} \FloatTok{0.45} \SpecialCharTok{+} \FloatTok{0.1428}\NormalTok{i}
\end{Highlighting}
\end{Shaded}

\subsection{Contexto - Editor de notação musical (EF15AR06, EF69AR22,
EM13ARH402)}\label{contexto---editor-de-notauxe7uxe3o-musical-ef15ar06-ef69ar22-em13arh402}

~~~~~~O exemplo a seguir ilustra os conceitos de altura (tonalidade) e
duração para notas musicais em escala diatônica. As legendas representam
os valores de duração, e as tonalidades são apresentadas em seus valores
de frequência

\href{Eq/jsp_musica2.html}{\pandocbounded{\includegraphics[keepaspectratio]{Eq/musica2.png}}}

\begin{Shaded}
\begin{Highlighting}[]
\FloatTok{1.}\NormalTok{ Experimente alterar a sequência melódica do código no vetor correspondente;}
\FloatTok{2.}\NormalTok{ Experimente alterar as figuras de duração no vetor correspondente}
\end{Highlighting}
\end{Shaded}

\section{Animação}\label{animauxe7uxe3o}

~~Assim como simulações, animações podem auxiliar na compreensão de um
tema específico. O \emph{JSPlotly} não trabalha diretamente com
animações como o faz a biblioteca \emph{Plotly.js}, pois isso exige
alteração do código-fonte que foge à proposta de ensino-aprendizagem e
pesquisa pretendidos (inserção de \emph{frames}, \emph{buttons},
\emph{transitions}, por ex). Não obstante, é possível simular uma
animação com vistas a melhorar o aprendizado de um tema, por inserção de
controle deslizante manual.

~~O exemplo a seguir apresenta um \emph{simulador para animação manual}
para funções matemáticas. Para seu uso, basta deslizar o \emph{slider}
progressivamente para visualização gráfica do resultado frente à mudança
da variável preditora.

\textbf{Obs:} Esse objeto didático tem um truque\ldots na verdade, dois
! Após clicar em \emph{add plo}, é necessário deslizar o \emph{slider}
primeiro para visualizar o gráfico. E para visualizar uma animação para
outra equação, é necessário atualizar a página, como orientado na margem
inferior da tela gráfica.

\href{Eq/jsp_anima2.html}{\pandocbounded{\includegraphics[keepaspectratio]{Eq/anima.png}}}

\subsection*{Sugestão:}\label{sugestuxe3o-15}
\addcontentsline{toc}{subsection}{Sugestão:}

\begin{Shaded}
\begin{Highlighting}[]
\FloatTok{1.}\NormalTok{ Deslize o controle para evidenciar a animação manual;}
\FloatTok{2.}\NormalTok{ Experimente substituir a equação modelo por outra, e arraste o controle deslizante para observar o efeito;}
\FloatTok{3.}\NormalTok{ Altere alguns parâmetros para a animação, por exemplo, aumentando os níveis de }\StringTok{"frames"}\NormalTok{ (}\AttributeTok{slider.max =} \StringTok{"50"}\NormalTok{; }\AttributeTok{slider.value =} \DecValTok{1}\NormalTok{).}
\end{Highlighting}
\end{Shaded}

\section{STEAM}\label{steam}

\subsection{Contexto - Vaso em torno de olaria (EM13MAT101, EM13MAT403,
EM13CNT204, EM13AR01,
EM13AR02)}\label{contexto---vaso-em-torno-de-olaria-em13mat101-em13mat403-em13cnt204-em13ar01-em13ar02}

~~~~~~A plataforma também permite criações para a integração em Ciência,
Tecnologia, Engenharia, Artes, e Matemática (\emph{STEAM}). Segue um
exemplo de simulação para torneamento cerâmico e moldagem manual de
argila, e que permite experimentar formas simétricas e arredondadas,
como vasos, tigelas e potes, por ajuste em alguns parâmetros e nas
funções trigonométricas do código.

\href{Eq/jsp_steam2.html}{\pandocbounded{\includegraphics[keepaspectratio]{Eq/steam.png}}}

\begin{Shaded}
\begin{Highlighting}[]
\FloatTok{1.}\NormalTok{ Altere a altura do vaso, seu formato, e suas cores, editando o código nos campos específicos.}
\end{Highlighting}
\end{Shaded}

\subsection{Contexto - Mini CAD (EM13MAT301, EM13MAT503,
EM13MAT402)}\label{contexto---mini-cad-em13mat301-em13mat503-em13mat402}

~~~~~~Segue um exemplo de código para manipulação de formas geométricas
em 3D (curvas ou retas) na elaboração de um mini CAD
(\emph{Computer-Aided Design}).

\href{Eq/jsp_miniCAD2.html}{\pandocbounded{\includegraphics[keepaspectratio]{Eq/miniCAD.pdf}}}

\begin{Shaded}
\begin{Highlighting}[]
\FloatTok{1.}\NormalTok{ Experimente alterar os parâmetros base, altura e curvatura do código, variando também o sinal dos }\FunctionTok{valores}\NormalTok{ (positivo, negativo);}
\FloatTok{2.}\NormalTok{ Altere alguma função trigonométrica (linhaX ou linhaY, seno para tangente, por exemplo), e sobreponha ao plot;}
\FloatTok{3.}\NormalTok{ Sobreponha figuras geométricas com paleta de cores distintas.}
\FloatTok{4.}\NormalTok{ Crie figuras simétricas sobrepondo uma curva com parâmetro positivo a uma com mesmo parâmetro negativo.}
\end{Highlighting}
\end{Shaded}

\section{Jogos}\label{jogos}

~~Em função do \emph{JSPlotly} envolver uma linguagem de programação,
\emph{JavaScript}, é plausível que consiga oferecer um conjunto de
operações tangíveis a essa, independentemente da construção de gráficos
(como em diagramas e fluxogramas acima).

~~Em paralelo à riqueza que acompanha a \emph{gameficação} como
metodologia ativa, a \emph{criação de um jogo} pode estimular o aprendiz
a um outro nível, posto que a estratégia por vezes já faz parte de seu
cotidiano. Jogar é uma coisa\ldots criar um jogo pode ter impacto mais
proeminente e recursivo ao pensamento e lógica computacionais, bem como
ao aprendizado da própria linguagem de programação !

\subsection*{Contexto - Jogo da Memória (EF06MA19, EF06MA16, EF07MA26,
EF09MA19, EM13MAT401, EM13MAT102,
EM13MAT403)}\label{contexto---jogo-da-memuxf3ria-ef06ma19-ef06ma16-ef07ma26-ef09ma19-em13mat401-em13mat102-em13mat403}
\addcontentsline{toc}{subsection}{Contexto - Jogo da Memória (EF06MA19,
EF06MA16, EF07MA26, EF09MA19, EM13MAT401, EM13MAT102, EM13MAT403)}

~~A imagem clicável abaixo direciona para um clássico \emph{Jogo de
Memória Numérica}. O objetivo é acertar um par de valores após
memorização de um quadro numérico. Para jogar, siga as instruções que
seguem:

\hfill\break

\textbf{Instruções para o Jogo de Memória Numérica:}

\begin{itemize}
\tightlist
\item
  \begin{enumerate}
  \def\labelenumi{\arabic{enumi}.}
  \tightlist
  \item
    No campo superior existem duas constantes booleanas true/false, uma
    para jogar (\emph{jogue}) e outra para verificar o acerto
    (\emph{verificar}), bem como uma \emph{semente} que fixa um
    determinado quadro numérico aleatório;
  \end{enumerate}
\item
  \begin{enumerate}
  \def\labelenumi{\arabic{enumi}.}
  \setcounter{enumi}{1}
  \tightlist
  \item
    Inicie o jogo (\emph{jogue/false}, \emph{verificar/false});
  \end{enumerate}
\item
  \begin{enumerate}
  \def\labelenumi{\arabic{enumi}.}
  \setcounter{enumi}{2}
  \tightlist
  \item
    Clique no botão \emph{add plot} e será exibido um quadro de pares
    numéricos para memorização;
  \end{enumerate}
\item
  \begin{enumerate}
  \def\labelenumi{\arabic{enumi}.}
  \setcounter{enumi}{3}
  \tightlist
  \item
    Troque para \emph{jogue/true}, clique em \emph{clean plot}, e depois
    em \emph{add plot}. O quadro será exibido agora apenas com um valor
    (demais células mostratrão \emph{``?''});
  \end{enumerate}
\item
  \begin{enumerate}
  \def\labelenumi{\arabic{enumi}.}
  \setcounter{enumi}{5}
  \tightlist
  \item
    Escolha \emph{verificar/true}, clique em \emph{clean plot}, e depois
    em \emph{add plot}. Será apresentado um campo para se digitar as
    coordenadas em que se acredita estar o outro par numérico (Ex: A2);
  \end{enumerate}
\item
  \begin{enumerate}
  \def\labelenumi{\arabic{enumi}.}
  \setcounter{enumi}{6}
  \tightlist
  \item
    Clique em \emph{OK}, e será apresentado um par formado pelo valor
    inicial e o valor escolhido, para verificação do acerto.
  \end{enumerate}
\end{itemize}

~~Para reiniciar o jogo, \emph{clean plot}, e opções booleanas
\emph{false}.

~~Divirta-se !!

\hfill\break

\href{Eq/jsp_game2.html}{\pandocbounded{\includegraphics[keepaspectratio]{Eq/game.png}}}

\begin{Shaded}
\begin{Highlighting}[]
\FloatTok{1.}\NormalTok{ Para jogar novamente alterando os valores do quadro, basta modificar a constante }\SpecialCharTok{*}\NormalTok{semente}\SpecialCharTok{*}\NormalTok{ para um número qualquer;}
\FloatTok{2.}\NormalTok{ Para variar entre níveis de dificuldade do jogo, basta alterar o número de linhas e de colunas nas constantes respectivas do código. }
\end{Highlighting}
\end{Shaded}

\href{workshop2.html}{Workshop - Funcionalidades diversas \& Exemplos
Adicionais: JSPlotly}




\end{document}
